% NOTE: Restricted length, must be between 150 and 500 words
The pervasiveness of the use of scientific models in all fields of research has led to model selection being a critical and challenging component of the scientific process for modern day research. Given a phenomena of interest, the model selection problem includes the following challenges: identifying which models exist, accessing the models in a form that allows for observation and experimentation, and comparing the models by some metric in order to determine which model is best suited for a given set of experimental criterion. Modern day researchers are required to overcome all of these challenges with little computational aides if they wish to be certain that the models they are using to observe phenomena do represent the best that the state-of-the-art research in their field of study has to offer. In this thesis I will present a computational tool that automates the process of extracting scientific models that are present in scientific source code, grounding the extracted models to ancillary documents, and efficiently analyzing the models both individually and comparatively to facilitate automatic model selection.
