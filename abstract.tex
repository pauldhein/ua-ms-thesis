% NOTE: Restricted length, must be between 150 and 500 words

Modern science is increasingly driven by the use of computational models of complex systems implemented in software.
This provides enormous opportunities, including increased precision in predictions, the possibility of better control over reproducibility of results, and the ability to aggregate models from different domains for larger-scale modeling.
However, realizing the full potential of computational scientific modeling requires addressing a number of new challenges.
A central challenge is that scientific model software is generally implemented in code bases that do not follow uniform conventions, making it so that model interpretation, selection and integration still requires very labor-intensive manual curation.
In this thesis I present a computational tool to assist in the process of extracting a representation of models from source code that permits grounding variables to domain concepts and performing model analysis.
To accomplish this task I will first define how a scientific model can be extracted from source code as a computation graph.
Using this computation graph I will show how scientific models can be compared via their shared structures and analyzed with existing methods of sensitivity analysis.
Finally the thesis will conclude with a discussion of how the information gained during model comparison and analysis can be used to inform model selection.
