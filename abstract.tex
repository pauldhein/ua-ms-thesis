% NOTE: Restricted length, must be between 150 and 500 words

\noindent Modern science is increasingly driven by the use of computational models of complex systems implemented in software.
This provides enormous opportunities, including increased precision in predictions, the possibility of better control over reproducibility of results, and the ability to aggregate models from different domains for larger-scale modeling.
However, realizing the full potential of computational scientific modeling requires addressing a number of new challenges.
A central challenge is that scientific model software is generally implemented in code bases that do not follow uniform conventions, making it so that model interpretation, selection and integration still requires very labor-intensive manual curation.
In this thesis I present the Unified Model Assembly Framework (UMAF), which aims to assist domain scientists with the process of extracting a representation of models from source code that permits grounding variables to domain concepts and performing model analysis.
To accomplish this task I will first define how information about a scientific model can be extracted from source code.
Using this information I will then show how an executable scientific model can be developed in the form of a dataflow program.
I will then show how this executable model can be used for the tasks of model analysis and comparison.
Finally the thesis will conclude with an evaluation of the current capabilities of the UMAF framework in terms of the amount of source code it can translate into dataflow-based scientific models and the accuracy of translation.
