\chapter{INTRODUCTION\label{chapter:introduction}}
Scientific models are increasingly expressed as executable software.
This enables a host of opportunities, including the following:
\begin{itemize}
\item increased precision in predictions,
\item the possibility of better control over reproducibility of results,
\item better communication of model details through unambiguous code implementation, and
\item the potential to aggregate individual models from different domains into larger multi-domain models.
\end{itemize}

These advantages have been the driving force that has pushed the scientific community towards computational models as a method of communicating scientific research.
To provide a working example, consider the computational models presented in Figure~\ref{fig:simple_crop_CAG}, expressed as \emph{causal analysis graphs} (CAGs), where nodes represent variables and directed arcs represent a functional relationship between variables (this formalism will be described in more detail in Chapter~\ref{chapter:extraction}).
These models describe how the yield of a particular crop is affected by changes in the amount of rain and soil water absorption rate over time (expressed in \emph{Day}s).

\begin{figure}[!htbp]
  \label{fig:simple_crop_CAG}
  \centering
  \tikz{ % Simple Crop Yield model example
    \tikzstyle{readable}=[rectangle, thick, rounded corners]
    \node[latent, readable] (crop_yield) {$Yield$} ; %
    \node[latent, readable, above=of crop_yield] (total_rain) {$Rain_{total}$} ; %
    \node[latent, readable, above=of total_rain] (rain) {$Rain$} ; %
    \node[obs, readable, above=of rain] (max_rain) {$Rain_{max}$} ; %
    \node[obs, readable, left=of max_rain] (absorption) {$Absorption$} ; %
    \node[obs, readable, right=of max_rain] (consistency) {$Consistency$} ; %
    \node[obs, readable, right=of rain] (day) {$Day$} ; %
    \edge {day, consistency, absorption, max_rain} {rain} ; %
    \edge {rain} {total_rain} ; %
    \path [->] (total_rain) edge  [loop right] (total_rain);
    \edge {total_rain} {crop_yield} ; %

    \plate {loop} {(rain)(day)(total_rain)} {$Day$} ;
  }
  \tikz{ % Different Crop Yield model example
    \tikzstyle{readable}=[rectangle, thick, rounded corners]
    \node[latent, readable] (crop_yield) {$Yield$} ; %
    \node[latent, readable, above=of crop_yield] (total_rain) {$Rain_{total}$} ; %
    \node[latent, readable, above=of total_rain] (rain) {$Rain$} ; %
    \node[obs, readable, above=of rain] (max_rain) {$Rain_{max}$} ; %
    \node[obs, readable, right=of max_rain] (absorption) {$Absorption$} ; %
    \node[obs, readable, left=of rain] (consistency) {$Consistency$} ; %
    \node[obs, readable, left=of total_rain] (sunlight) {$Sunlight$} ; %
    \node[obs, readable, right=of rain] (day) {$Day$} ; %
    \edge {day, absorption, max_rain} {rain} ; %
    \edge {rain, consistency} {total_rain} ; %
    \path [->] (total_rain) edge  [loop right] (total_rain);
    \edge {total_rain, sunlight} {crop_yield} ; %

    \plate {loop} {(rain)(day)(total_rain)} {$Day$} ;
  }
  \caption[Competing Models of Crop Yield]{Two competing scientific models depicting the affects of rain on the yield of a crop over a span of days given some other inputs. We see that the two models share many of their inputs but that some inputs may not be shared and the wiring of the inputs to the output variable can differ between the models.}
\end{figure}

Both of these models have sets of input values, shaded to indicate
that they are observed, and output values. Upon observing the inputs through measurement, a value for the output can be computed.
The models have some shared inputs between them and some similar structure that leads to the same output variable in each model.
However there are differences between these two models, namely that one has an additional input and that the internal wirings of the variables in the two models differ.
The models could differ even more than this, in fact it is possible for competing models to only share the same output node.
Modelers will be particularly interested in comparing models that at least share the same output node, since they will want to select one model amongst those of the same phenomena that performs the best for their studies.

Performing comparisons between competing models requires the ability to identify how different models utilize information from their inputs.
For models that are defined in source code, an easy way to identify these differences is by comparing and analyzing the models computation graphs.
Since most competing models that exist in source code are likely written in different programming languages, any modeler who wants to compare and analyze the computation graphs of competing models will need to perform a translation task that will translate all of the competing models into a single representation language.
This representation language needs to enhance the representation of models found in the source code by associating the variables found in the source code models to the measured real-world phenomena that they represent.
Accomplishing this would allow the shared variables between competing models defined in source code to be identified.
After completing these tasks a modeler would then be able to undertake the task of performing comparison and analysis between the competing models.

\section{Problem Scope\label{sec:prob_scope}}
This thesis describes a set of related data structures and algorithms that are part of a framework aimed at automating parts of the task of analyzing scientific models implemented in software. In particular, this thesis addresses the following four tasks underlying automated computational model analysis and selection:

\begin{enumerate}
  \item A processing pipeline to extract models from scientific source code and represent them in a source programming language-agnostic representation that supports general model analysis.
  \item An algorithm to identify model structural overlap, as a basis for identifying how models are structurally similar and different.
  \item An algorithm for searching for input value ranges that make overlapping models behave as close to each other as possible, and thereby also identify value ranges that distinguish models (a basis for experiment design).
  \item Adaptation of sensitivity analysis measures to identify model output sensitivity as a function of model input value ranges within a uniform analysis framework.
\end{enumerate}

Combining these capabilities into a single framework provides a facility for domain-expert model developers and analysts to now analyze and compare models within a uniform framework, greatly simplifying model analysis tasks that to-date have required enormous manual effort.
This system does not take the human out of the loop: domain expertise and human guidance are still needed to identify variable value ranges of interest to a modeling application, as well as supplement mistakes of omission and commission that may be made during variable grounding.
Ongoing and future development is also required to scale the methods to effectively handle larger model code bases.
However, this framework described here is, to our knowledge, the first general approach to automating aspects of model analysis in the support of general model comparison and selection.
We also believe these tools provide a basis for a new kind of model curation and debugging, allowing one to compare changes within evolving code bases but from a modeling domain-semantics perspective. Exploring use of this framework for this purpose will be the subject of future work.

\section{Related Work\label{sec:prior_work}}
For modelers in many disciplines the current state-of-the art for addressing the problem of model selection is not to use an intelligent tool or service that performs the selection, but to dig into the literature and find examples of model comparisons that have already been done amongst competing models in a given application domain.
Unsurprisingly, this approach has many disadvantages.
The obvious initial cost of this approach is that it requires a group of scientists to devote a substantial portion of time to conducting this analysis by hand.
A second disadvantage of this method is that the analysis performed and documented by research groups in scientific publications is difficult to extend.
For example the analysis conducted by \citet{camargo2016six} contains information on the sensitivity of the compared models to certain inputs; however, modelers are left without any discussion or investigation into the pairwise affect of inputs on model sensitivity.
In order for modelers to obtain this information, their best option is to request access to the software used by the authors so that they can manually extend the software to extend the analysis.
This is a less-than-ideal solution as not only will additional time be required to extend the analysis, but modelers must now rely upon the generosity of the authors to release the software they used to conduct their analysis, otherwise extending the analysis presented would require conducting all of the software design necessary to produce the initial analysis.
Therefore, it is easy to see how automating the process of model extraction, analysis, and selection would greatly benefit modelers by increasing the extensibility of an analysis.

Current tools such as the DAKOTA system make advances towards a generalized framework that provides modelers analysis tools that can be used to study and compare models \citep{adams2009dakota}.
DAKOTA has been an excellent resource for modelers because of the many evaluation methods that are provided for model study.
However DAKOTA has some disadvantages.
The most notable disadvantage of the DAKOTA system is that modelers are still required to undergo the labor-intensive task of translating their models by hand into the pre-defined DAKOTA format.
Not only does the labor and time present a barrier to entry for modelers wishing to use the DAKOTA system, but this system comes with an additional barrier for modelers who wish to use models from other researchers.
While the modelers will likely be familiar with the structure of their own models, allowing for a simple translation task, they will likely require additional time and mental labor to become well-acquainted with model specifications developed for the models in other software systems in order to create a faithful translation of the groups model to the required DAKOTA format.
Adaptation or reimplementation of other code bases introduces an unnecessary potential source of error.
Another shortcoming of the DAKOTA system is that it does not provide any service to \emph{compare} models.
Modelers may be interested in what variables overlap between competing models, and may wish to augment the existing models to allow for analysis upon the overlapped portions of two competing models.
DAKOTA does not offer any approach to help with this task.
This means that while DAKOTA is a very useful tool for modelers to analyze existing models that they are well-acquainted enough to translate, it does not provide help in performing larger scale model selection, as modelers are still required to do large amounts of setup, reconstruction, and investigation of analysis methods in order to gather useful information from DAKOTA that they can then use to select a model.

New methods have been created in recent years that are beginning to automate the process of extracting scientific models from free text.
One such method is presented in the Eidos, INDRA, and Delphi \citep{EidosIndraDelphi} pipeline that seeks to extract models from textual documents, such as scientific publications and technical reports.
This software has seen great success at the task of model extraction and assembly.
Indeed many models are presented in the scientific literature and sometimes only in the scientific literature without being present in the form of source code in a software repository.
However modelers will undoubtedly benefit from the reliability guarantees of models that are extracted from source code.
Due to the nature of free text, identifying and extracting models from free text is very challenging: authors are not forced to write effective procedures, instead describing models at higher levels of descriptions and often eliding information assumed as common-sense (or domain-specific) background knowledge.
Source code, on the other hand, is explicit.
Source code also requires full specification in order to be executable, so models extracted from source code may include background assumptions that are left out of textual descriptions.
A simple example of this phenomena would be the lack of data type information for numerical values. This information is commonly absent from the text descriptions of a model, but is vital in many cases when creating an executable version of the model.

However, extracting models from source code does share at least one of the challenges associated with extracting models from free text:
associating or \emph{grounding} the variables present in the source code to the aspects of the domain being modeled.
The approach presented in this thesis makes use of the approach of the Eidos, INDRA, and Delphi system to ground textual mentions of concepts with domain taxonomies.

\section{This Work\label{sec:this_work}}
In this thesis I will present the Scientific Model Selection (SMS) pipeline: a system that provides a set of tools to help with the comparison and selection of models extracted from source code.
To accomplish this task the system must be able to extract scientific models from source code, ground the real-world variables contained in the models using information gained from associated texts, and finally perform the model selection task using information gained from sensitivity analysis.
The SMS pipeline is a component of the larger software system designed as part of the DARPA funded Automated Model Assembly from Text Equations and Software \citep{pyarelal2019} (AutoMATES\footnote{\url{https://ml4ai.github.io/automates/}}) project.

As a component of the AutoMATES project, this pipeline focuses on part of the overall goal of model extraction, grounding, and analysis.
The AutoMATES project includes three additional modules that all provide input to the SMS pipeline.
These modules are the Program Analysis (PA) pipeline, the Text Reading (TR) module, and the Equation Reading (ER) module.
The inputs provided by these modules will assist the SMS pipeline in extracting abstract representations of source code from the original source code languages, as well as grounding the variables found in that source code to real-world concepts.
Any inputs to the SMS pipeline from these modules will be documented in the sections of the thesis where they are used.

The remainder of this thesis is organized to present the components of the pipeline that will solve the problem presented in the problem scope section.
During the course of this thesis I will use two models from the Decision Support System for Agrotechnology Transfer (DSSAT)\footnote{\url{https://dssat.net/}} software system \citep{DSSAT}.
% model collection to demonstrate the capabilities of the SMS pipeline.
The models used in this study will be targeting the natural phenomena of Potential Evapo-Transpiration (PET).
The specific models I will be comparing are the Priestly-Taylor model of Potential Evapo-Transpiration (PETPT) and the ASCE model of Potential Evapo-Transpiration (PETASCE).
Chapter~\ref{chapter:extraction} will introduce the algorithms used to extract models from source code and transform them into a form that is both executable and comparable across competing models.
Chapter~\ref{chapter:extraction} will also introduce the analysis methods used by the SMS pipeline to derive information about the models under comparison.
Chapter~\ref{chapter:analysis} will document how the derived information from the analysis phase can be used to perform automated model selection, as well as how the information will be presented to users of the SMS pipeline to allow them to make their own final model selection decisions.
This thesis will then conclude with Chapter~\ref{sec:conclusions} that will be a discussion of the results and implications of the pipeline and will introduce possible extensions for continuing this research.
