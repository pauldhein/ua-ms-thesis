\chapter{INTRODUCTION\label{chapter:introduction}}
When studying a given natural phenomena, many researchers turn to modeling as a method to formalize their reasoning about the phenomena. Models have many advantages for researchers, including:

$1.)$ They allow researchers to clearly communicate the relations between variables observed to be associated with a phenomena. This gives the researchers the ability to reason about uncertainty within a model while incorporating information from past experiments as background knowledge.

$2.)$ They are exportable, comparable, and updatable. One researcher can use the model of another, competing models can be compared, and under-performing components of a model can be updated upon discovering new information about the model.

These advantages have been the driving force that has pushed the scientific community towards models as a method of communication of scientific research. However, modeling is not without its challenges. One of the new issues facing scientific researchers is the sheer prevalence of models. Model selection is now a task facing modelers in any field of research. Not only do scientists have to explore many competing models when deciding which to use to model a particular phenomena, this exploration is also expensive. In the information age, most of these models exist as source code with associated grounding documents. However, with the extreme prevalence of programming languages many competing models for the same phenomena are likely written in different programming languages. Asking scientist to learn a single programming language is already a large drain on research time, but the prospect of needing to learn multiple languages represents a large barrier to entering the realm of model selection. Given the enormous amount of items competing for the limited time of scientists the task of model selection commonly is side-lined.

In this thesis I will present an automated system that is able to extract scientific models from source code, ground the models using information gained from associated texts, and finally automate the task of model selection. Accomplishing this task will further unlock the potential of models to revolutionize the objective study of naturally occurring phenomena.

% NOTE: previous head of intro that needs to be better tied together, possibly with sections
Below we see an example of a model being used to study the yield of crops by observing changes in multiple input variables.


% Simple Crop Yield model example
\begin{figure}[ht]
  \begin{center}
    \begin{tabular}{cc}
      \tikz{ %
        \tikzstyle{readable}=[rectangle, thick, rounded corners]
        \node[latent, readable] (crop_yield) {$Yield$} ; %
        \node[latent, readable, above=of crop_yield] (total_rain) {$Rain_{total}$} ; %
        \node[latent, readable, above=of total_rain] (rain) {$Rain$} ; %
        \node[obs, readable, above=of rain] (max_rain) {$Rain_{max}$} ; %
        \node[obs, readable, left=of max_rain] (absorption) {$Absorption$} ; %
        \node[obs, readable, right=of max_rain] (consistency) {$Consistency$} ; %
        \node[obs, readable, right=of rain] (day) {$Day$} ; %
        \edge {day, consistency, absorption, max_rain} {rain} ; %
        \edge {rain} {total_rain} ; %
        \path [->] (total_rain) edge  [loop right] (total_rain);
        \edge {total_rain} {crop_yield} ; %

        \plate {loop} {(rain)(day)(total_rain)} {$Day$} ;
      }
    \end{tabular}
  \end{center}
  \caption[Crop yield model]{A model depicting the affects of rain over a span of days given observed values for absorption and consistency constants.}
\end{figure}

\section{Problem Definition\label{sec:problem_def}}

Some text.


\section{This Work\label{sec:this_work}}

Some text.
