\chapter{INTRODUCTION\label{chapter:introduction}}

Below we see an example of a model being used to study the yield of crops by observing changes in multiple input variables.


% Simple Crop Yield model example
\begin{figure}[ht]
  \begin{center}
    \begin{tabular}{cc}
      \tikz{ %
        \tikzstyle{readable}=[rectangle, thick, rounded corners]
        \node[latent, readable] (crop_yield) {$Yield$} ; %
        \node[latent, readable, above=of crop_yield] (total_rain) {$Rain_{total}$} ; %
        \node[latent, readable, above=of total_rain] (rain) {$Rain$} ; %
        \node[obs, readable, above=of rain] (max_rain) {$Rain_{max}$} ; %
        \node[obs, readable, left=of max_rain] (absorption) {$Absorption$} ; %
        \node[obs, readable, right=of max_rain] (consistency) {$Consistency$} ; %
        \node[obs, readable, right=of rain] (day) {$Day$} ; %
        \edge {day, consistency, absorption, max_rain} {rain} ; %
        \edge {rain} {total_rain} ; %
        \path [->] (total_rain) edge  [loop right] (total_rain);
        \edge {total_rain} {crop_yield} ; %

        \plate {loop} {(rain)(day)(total_rain)} {$Day$} ;
      }
    \end{tabular}
  \end{center}
  \caption[Crop yield model]{A model depicting the affects of rain over a span of days given observed values for absorption and consistency constants.}
\end{figure}

\section{Problem Definition\label{sec:problem_def}}

Some text.


\section{This Work\label{sec:this_work}}

Some text.
