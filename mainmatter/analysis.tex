\chapter{AUTOMATED MODEL ANALYSIS\label{chapter:analysis}}

Some intro about the task of analyzing two or more GrFN models that have been extracted from source code, or the Forward Influence Blanket of two or more GrFNs.

\section{Sampling Techniques\label{sec:samp_overview}}

Some text about the basic method for sampling from the inputs to a GrFN given little information.

\subsection{Saltelli Sampling\label{sec:saltelli_samp}}

This section should introduce Saltelli sampling and go into great detail on the process.

\subsection{Data Informed Sampling Techniques\label{sec:data_samp}}

This section should introduce the methods that we can use to sample from pre-existing data files associated with the codebase that we are working with.

\subsection{Grounded Sampling Techniques\label{sec:grounded_samp}}

This section should go into detail on how we can use the grounded nature of our variables to get information about how they exists in the real world that we can use to constrain the domains for our inputs.

\section{Sensitivity Analysis\label{sec:sensitivity_overview}}

Some text that discusses our methods for conducting Sensitivity analysis of our extracted GrFNs.

\subsection{Variance Based Analysis\label{sec:sobol_sensitivity}}

This section will go in depth on the Sobol method of Sensitivity analysis.


\section{Output Surface Analysis\label{sec:surface_analysis}}

This section will introduce the idea of generating output surfaces that scientists can view and interact with in order to better understand the sensitivity of their models based upon key inputs. The choice in which surfaces to show will be directed by sensitivity analysis.
