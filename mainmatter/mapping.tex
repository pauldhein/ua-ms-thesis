\chapter{MODEL EXTRACTION FROM SOURCE CODE\label{chapter:extraction}}
In order to perform analysis on scientific models found in software we must first build a system that is able to detect source code that corresponds to scientific models, extract that necessary associated source code, and then represent the model in a form that allows analysis to take place. The technique developed must be generalizable to allow model extraction from source code of the various programming languages that are used to define scientific models. While large variations in syntax exist among the various programming languages used to encode scientific models, they can all be abstracted to a unifying abstract syntax tree (AST) representation. Constructing an AST representation of a program requires a large number of design decisions that are largely dependent on the intended use-case of the information contained in the AST. For the purposes of model selection we are keenly interested in identifying all potential models contained in the AST, as well as being able to easily inspect data-flow through the AST.

This chapter of this thesis will begin with a discussion of the AST representation design decisions made to facilitate model extraction from source code. Following this discussion will be an explanation on the conversion process from the AST representation of a model to an executable computation graph. This chapter will conclude with an analysis of the computational cost of running the computation graph for a model and means of reducing this cost by leveraging vectorized computation and GPU computing resources.

At the time of writing this thesis the program analysis component of our AutoMATES system is capable of handling source code from the Fortran programming language. The AutoMATES team plans to support additional widely used languages in the near future and for the purposes of model analysis, and the extent of this thesis, the subtle language differences that will be introduced from different programming languages will be handled by the program analysis module, before being generated into AST form.


\section{Grounded Function Network Extraction\label{sec:grfn_extract}}
The SMS pipeline will receive a JSON specification and a set of associated lambda functions as input. The JSON specification will fully represent the control-flow structure of the original source code, while the lambdas file will contain all of the computations necessary to execute a model that is faithful to the models found in the original source code. From these two files the SMS pipeline will construct an executable model that can be used for comparison and analysis.

This executable model is called a Grounded Function Network (GrFN). As the name suggests, this model will be a network. A network is a directed acyclic graph that contains a set of source nodes and a set of sink nodes. A node is considered a source node if the in-degree of the node, the number of directed edges arriving at the node, is zero. Similarly a node is considered a sink node if the out-degree of the node, the number of directed edges leaving the node, is zero. Since our GrFN is supposed to represent a scientific model, the source nodes of the GrFN will be the inputs of the model represented by the GrFN, and the sink node will be the model output. The name GrFN also implies that this network will be a network with functions. Indeed this network will include two types of nodes, variable nodes and function nodes. This network will be bipartite such that no function node has an edge incident to another function node, and no variable node has an edge incident to another variable node. The last element left in the name has to deal with the naming of the variable nodes. The GrFNs must be comparable upon their variable nodes, since the variables track the actual observed and calculated values. This is what connects the networks to the real-world and to one another. As defined in code, two variables that represent the same phenomena can have different names, and two variables that represent different phenomena can have the same name. To circumvent this problem, the names of variable nodes in a GrFN will not come directly from the source code, but they will instead be shared names that are derived from grounding the variables found in source code with associated publications and other free text that can provide descriptive names for the variables contained in them. The grounding process ensures that variable node names are identical across various GrFNs, even if extracted from separate code repositories. This will allow for graph matching upon named nodes during downstream analysis.

\section{Computation Graph Generation\label{sec:cg_gen}}
Although the program analysis module is tasked with lifting an AST from the source code, the SMS pipeline is responsible for deducing scientific models from the AST and transforming them into GrFNs. As described in the section above, the program analysis module will lift the control-flow found in the AST into a JSON specification. The JSON specification has a finite set of statements that completely describe its contents, and developing rules for how to traverse the JSON and handle the included statements will provide a clear view of how a GrFN can be constructed from the JSON.


\subsection{Containers and Function Calls\label{sec:containers}}
At the top level the JSON has a list of containers. These containers represent different scope levels in the source program. A container can be a function or subroutine found in the source code or a loop. The branches of a conditional will not be considered containers due to the nature of how the GrFN will be wired to deal with conditional evaluation. a container will contain a body that will be in the form of an ordered list of statements. These statements will correspond to statements from the original source code that was present in the container. Computation graph construction will begin at a top-level container and will process through the statements in the container, constructing the computation graph as it goes. When a statement that references a new container, such as a function call, is reached the body for that statement will be processed and connected to the computation graph that is being generated. Once this task is completed, the rest of the body for the original container will be processed. This recursive process works well for traversing all containers and constructing a computation graph based upon the statements contained in each container as long as no container has a call in it's body to another container that also has a call to the original container. Unfortunately this is precisely the case with recursive functions. They act as a special case for this processing pipeline and thus they are handled separately as will be discussed in subsequent sections.

Landing on a function call when parsing the body of a container can be thought of as an indicator to process the body of the child container referenced by the function call. The only intricacy here deals with the correct wiring of variable inputs into the containers plate, and the correct wiring of outputs from the container to the current position in the computation graph.

\subsection{Assignment Statements\label{sec:assg_stmts}}
During the processing of a container, when an assignment statement is reached it will include the following items:
\begin{itemize}
  \item A list of source variables
  \item The name of a lambda function
  \item A target variable node
\end{itemize}
Using these items the wiring for an assignment statement performs the following tasks to fully incorporate the information contained in the assignment statement into the computation graph:
\begin{itemize}
  \item Create a new function node for the lambda function
  \item Create a new variable node for target variable
  \item Create new variable nodes as needed for the input variables
  \item Create a directed edge from each input variable node to the function node
  \item Create a directed edge from the function node to the target variable node.
\end{itemize}
Assignment statements can be handled by loading the assignment statement found in the source code into the function node so that values can propagate from the input variable nodes to the output variable node during computation graph execution.

\subsection{Conditional Statements\label{sec:cond_stmts}}
Conditional statements are handled via a set of two lambda evaluations. The first evaluation is known as a \texttt{conditional} function node, that will actually evaluate the conditional property. The second is a \texttt{decision} function node that takes as input the evaluation from the \texttt{condition} function as well as the two possible assignments for an output variable. The \texttt{decision} function will be responsible for assigning the appropriate value to the output variable node based upon the conditional input.

Both the \texttt{decision} function node and the \texttt{conditional} function nodes output variable node are artifacts that did not exist in the original source program that have been added to the computation graph. Thus these will not be displayed when rendering the function node or variable node views of the computation graph.

\subsection{Indexed and Open-ended Loops\label{sec:loops}}
Indexed loops require a loop plate and have a specific index variable as well as a number of iterations through the loop. They can easily be handled like containers as mentioned above, but require additional storage to handle information about the number of executions needed to satisfy the plate during compuation.

Open ended loops are hard. Here we can have conditional exit cases defined at a start or end point of a loop, which presents a much larger challenge than loops with an index and pre-defined amount of iterations.

An extra challenge is added when dealing with open-ended loops that can include multiple exit points (introduced either by \texttt{break} statements or \texttt{goto}s) as well conditional skip points where parts of the loop are skipped on an iteration (introduced either by \texttt{continue} statements or \texttt{goto}s).

% NOTE: section about loops from unstructured branching
The usage of the \texttt{goto} statement has been hotly debated by computer scientists for nearly half a century. In most modern programming languages the usage of  \texttt{goto} or other such statements that allow for unstructured branching is prohibited. However, the AutoMATES system seeks to handle source code inputs from languages that do allow for unstructured branching, and thus this paradigm must be handled during the wiring phase of a GrFN computation graph.

% NOTE: part about recursive functions as loops
Recursion is a commonly used software practice that must be handled for our computation graphs. Most importantly, recursion must be identified and recursive edges that would create loops in the computation graph must be pruned.

Possibly the most difficult challenge for our graph wiring is the identification and handling of indirect recursion.


\section{Data Type Assignment\label{sec:data_types}}
So far I have discussed the wiring needed to create the computation graph that will allow a GrFN to be executed, and I have shown the methods necessary to make the GrFN executable. However, one more crucial component for creating a GrFN that we can perform inference upon is a discussion of how we will handle the actual data being processed from the GrFN. At the time of writing this thesis only basic data types are allowable in a GrFN. This includes numerical, string, and boolean values. These primitive data types are all singular values that represent a single phenomena, thus they match perfectly with the definition of variable nodes in the GrFN CG. The specification for each variable in a GrFN CG contains a type annotation that declares the data type of the variable. These values are derived from the JSON specification during the wiring stage of the GrFN. At execution time, these type annotations are used to validate a set of inputs and ensure proper storage format of computed variable values.

The infrastructure to represent complex data types such as Arrays, user-defined types, and unions in the AST form is still being developed by the program analysis team, and thus they will not be included in this thesis. The main challenge with representing these data types is that they are not singular variables, but are instead collections of variables. To properly perform inference over a GrFN CG all variable nodes must be singular variables, thus we cannot allow any of the above collections of variables to be represented by a variable node. This presents a problem of representation that will be studied  and resolved in future work that extends this thesis.


\section{GrFN Computation Graph Execution\label{sec:cg_execution}}
Once the wiring stage for a GrFN has been completed, the GrFN CG must be made executable. The execution of a GrFN CG requires the execution of all functions stored at the function nodes of the computation graph. Therefore the problem of executing a GrFN CG can be simplified to the problem of determining an ordering of execution of the function nodes that allows for all of the function nodes to be executed without any failures. Of course a GrFN CG also needs to save the state of variables during execution in a way that allows the function nodes to access the information stored for each variable they require. In this section I will present the methods developed for the GrFN CG to handle the function evaluation ordering problem and the variable storage access problem.

\subsection{Call Stack Creation\label{sec:call_stack}}
Creating a computation graph from a GrFN specification allows us to formally represent an extracted scientific model as a graph data structure. However, if we wish to analyze the extracted model, then we will need the ability to compute information over this data structure. To accomplish this we introduce the idea of execution over a computation graph. The computation graph contains a set of function nodes. Computing the lambda function stored at each function node is analogous to executing the computation graph from the set of inputs to the output. However, the function nodes rely upon having values populated at each of their input variable nodes in order to perform their computation. Therefore the task of executing a GrFN CG can be rephrased as determining how to order and execute the functions nodes contained in the computation graph.

A Naïve first-pass solution to accomplish this goal would be to use a graph traversal from the output to the inputs where at each function node, the node will determine whether values for each input variable node have been populated. For any input variable nodes that have not been populated, the function node will call the parent function node responsible for computing the value of the input variable node. Once all such calls have returned, the function itself will evaluate. This recursive calling procedure is very similar to message-passing, a method for inference on factor graphs. While this will ensure correct model execution, this method of handling execution is not as efficient as possible. To start the recursive call structure adds additional function setup and calls to the execution, on the order of the number of functions included in the computation graph.

The computation graph has the form of a factor graph with variable and function nodes, such that no variable node is adjacent to another variable node and vice-versa for a function node. Therefore representing a computation graph in terms of just the contained function nodes

\subsection{Input Set Execution\label{sec:input_execution}}
During execution, a GrFN utilizes a value storage tag at each variable node. During computation, function nodes pull their input data from the value tag of each parent variable node of the function node. The output from the function node is stored in the value tag of the variable node that is the child of the function node.

\subsection{Vectorized Input Execution\label{sec:vector_input_execution}}
While a GrFN CG is perfectly capable of executing one set of inputs at a time, the CG can also handle executing over multiple sets of inputs at once. The SMS pipeline accomplishes this by making use of the PyTorch tensor computation framework. It is advantageous to compute multiple inputs at once using vectorized computations because pooling like computations lowers overall compute time.
