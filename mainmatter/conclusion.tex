\chapter{CONCLUSIONS AND FUTURE WORK\label{chapter:conc_and_future}}

% \section{Conclusions\label{sec:conclusions}}

\ctm{The Third Industrial Revolution (the information, computer or digital revolution) in the latter half of the 20th century has had a profound impact on how science is conducted. Large-scale scientific progress today relies marshaling computational large-scale science has become increasingly dependent on computational models}

% CTM Original:
%The problem of scientific model selection has become increasingly more important to scientists in recent years, and at the same time has increased in difficulty.
%This increase in difficulty is partially attributable to the increase in the number of models available, but a large part of the difficulty comes from the difficulty in gaining access to published models.
%Restricting the search space of possible models to only those defined in source code still leaves researchers with many time consuming, tedious tasks in order to select which model best fits their needs.
%These tasks include translating the models to a single source code language and common variable schema, validating the models with associated texts, analyzing the models with the correct statistical tools, and finally developing a criterion that determines which model is best for a given experiment.

The SMS pipeline, as a component of the AutoMATES project \citep{pyarelal2019}, introduces a framework that automates a portion of the
% tasks described above.
The work described in this thesis has est

Namely the task of translation to a singular representation of models as a GrFN has been completed, as well as the implementation of analysis methods that allow for critical comparative insights to be gathered about models that allow for a selection to be made.
The framework has been constructed in a manner that allows researchers to easily query results from the analysis of competing models when developing a criterion for selecting one model over another.
Additionally this framework is open to extension that will allow more of the tasks outlined above to be automated in future iterations.

\section{Future Work\label{sec:future_work}}
% CTM Original: As discussed above,
\ctm{The} AutoMATES project \ctm{is developing a}
% CTM Original: has created an excellent
framework for the extraction and comparison of scientific models found in source code.
While many methods for analysis already exist in the AutoMATES system, as well as information necessary to facilitate automated model selection, there are \ctm{a number} % still plenty
of directions for future work.
Many of the opportunities for extending the AutoMATES system build upon one another, and all are focused on expanding the scope of modeler questions that AutoMATES is able to handle without additional input from the modeler.
\ctm{In this section, I describe several extensions that can be made to the AutoMATES model analysis }
Below I catalog some of the immediately visible extensions to the AutoMATES program improve the power of the AutoMATES model selection capabilities.

% TODO: Add these back in if I have time and Clay approves
% \subsection{Alternative Sampling Methods for Analysis\label{sec:alt_sampling}}
% % TODO: Fill in here if I have time.
% Some text.
%
% \subsection{Variable Domain and Range Detection\label{sec:var_domain_range}}
% % TODO: Consider moving this section to future work if we run out of time to implement a solution to this problem
% Some text.

\subsection{Model Selection via Uncertainty Analysis \label{auto_uncert_analysis}}
The current analysis methods employed by AutoMATES allow for automated model selection based upon behavior of model inputs or sets of model inputs. While these methods are likely desired by modelers, they only allow for indirect model comparisons. A method for direct comparison such as error propagation that includes an estimate of metrics such as variance in model output allows for stronger comparison statements that will likely be more acceptable metrics for automated model selection.

\subsection{Iterative Model Improvement\label{sec:auto_improve}}
After gathering enough information about various competing models as well as error information of model improvements,AutoMATES should be able to begin learning how to update models to lower uncertainty in model outputs.

A key component to this enhancement would be to identify similar function nodes or series of function nodes in a computation graph that correspond to the same overall computation. This, along with the grounding of variable nodes, which has been assumed, will enable modular computation components for variables to be added, removed, or mutated in order to improve model accuracy, efficiency, or other metrics of choice to modelers.

\subsection{Input Space Division\label{sec:auto_isd}}
Modelers would likely benefit from the AutoMATES being able to answer more general questions about what models to use to study a certain phenomena. For instance, modelers may not be able to provide bound information for the variables of interest to them when gathering data to study a particular phenomena. AutoMATES could assist modelers in this regard by discovering the furthest possible extent of all possible variables for each competing model of a phenomena and then partition the input variable space based upon peak model performance, such that each separate partition has an identified ideal model for studying the phenomena given inputs contained in that range. As previously stated the partition criterion would be some aspect of model fitness.

\subsection{Data Space Discovery\label{sec:auto_dsd}}
An extension of the idea of automatically discovering the input bounds of a set of input variables for a model is the idea of discovering the total possible data space for a phenomena of interest. Modelers would benefit from future versions of AutoMATES being able to identify potential additional variables for a given phenomena, other than just those identified by the modeler. A potential example would be the identification of a combination of variables that can be used to model a given model input with higher precision.
